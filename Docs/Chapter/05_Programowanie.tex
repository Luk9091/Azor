\section{Programowanie}
    \tab Do wykorzystania przez użytkownika został oddany jeszcze jeden interfejs kontrolowania „Azora”.
    Jest nim prosty język służy do programowania prostych algorytmów, które Azor ma wykonywać.
    Aby zaprogramować „Azora” należy zamknąć okno interfejsu graficznego oraz zakończyć połączenie z „Azorem”. 
    Następnie należy przejść do folderu „Assembler”, w którym znajdują się dwa skrypty: „compile” oraz „programming”.
    Pierwszy skrypt służy do kompilacji mnemonicznych instrukcji na język wartości zrozumiałe przez interpreter Azora.
    Drugi skrypt wykorzystywany jest do wgrywania programu oraz sprawdzenie czy program wgrał się poprawnie.

    W celu stworzenia własnego programu należy, stworzyć plik tekstowy. Napisać program z pomocą instrukcji zamieszczonych w tabeli poniżej, skompilować go za pomocą komendy:
    \begin{lstlisting}[gobble = 8, frame = L]
        $ python compile.py <nazwa pliku>
    \end{lstlisting}
    oraz wgrać go do Azora za pomocą komendy:
    \begin{lstlisting}[gobble = 8, frame = L]
        $ python programming.py <nazwa pliku>.dec
    \end{lstlisting}

    \begin{table}[!h]
        \centering
        \begin{tabularx}{\textwidth}{|l|l|>{\centering\arraybackslash}X|>{\centering\arraybackslash}p{1.5cm}|}\hline
            Nr. & Instrukcja & Opis & Ilość bajtów\\\hline
             1. & NOP & Nic nie rób & 1\\\hline
             2. & END & Zakończ wykonywanie programu & 1\\\hline
             3. & SLEEP & Usypia procesor i wyłącza wszystkie peryferia* & 1 \\\hline
             4. & SET $<reg>$ & Ustaw wskaźnik adresu pamięci & 1 \\\hline
             5. & I2C\_WRITE $<reg>$ & Zapisz do pamięci EEPROM wartość & 1 \\\hline
        \end{tabularx}
        \caption{Tabela instrukcji „Azora”}
    \end{table}