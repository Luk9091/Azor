\section{Programowanie}
    \tab Do wykorzystania przez użytkownika został oddany jeszcze jeden interfejs kontrolowania „Azora”.
    Jest nim prosty język służy do programowania prostych algorytmów, które Azor ma wykonywać.
    Aby zaprogramować „Azora” należy zamknąć okno interfejsu graficznego oraz zakończyć połączenie z „Azorem”. 
    Następnie należy przejść do folderu „Assembler”, w którym znajdują się dwa skrypty: „compile” oraz „programming”.
    Pierwszy skrypt służy do kompilacji mnemonicznych instrukcji na język wartości zrozumiałe przez interpreter Azora.
    Drugi skrypt wykorzystywany jest do wgrywania programu oraz sprawdzenie czy program wgrał się poprawnie.

    W celu stworzenia własnego programu należy, stworzyć plik tekstowy. Napisać program z pomocą instrukcji zamieszczonych w tabeli poniżej, skompilować go za pomocą komendy:
    \begin{lstlisting}[gobble = 8, frame = L]
        $ python compile.py <nazwa pliku>
    \end{lstlisting}
    oraz wgrać go do Azora za pomocą komendy:
    \begin{lstlisting}[gobble = 8, frame = L]
        $ python programming.py <nazwa pliku>.dec
    \end{lstlisting}


    \subsection{Rejestry}
        \tab Do dowolnego użytku, zostały oddane cztery 16bitowe rejestry: REG0, REG1, REG2, REG3.
        Każdy z podanych rejestrów może zostać w dowolnej instrukcji.
        To znaczy, że każdego rejestru można wykorzystać jako akumulatora, zewnętrznego czy rejestru przesuwnego.
        Znaczną wadą całego układu jest jednak brak możliwości swobodnej migracji danych między rejestrami, autorzy proponują rozwiązanie przedstawione poniżej:
        \begin{lstlisting}[gobble = 8, frame = L, caption={Skopiowanie wartości z Rejestru 0 do Rejestru 1}, captionpos=b]
            PUSH R0 ; zapisanie na stosie wartości z R0
            POP R1  ; Pobranie ze stosu wartości R0 i zapisanie jej w R1
        \end{lstlisting}

    \subsection{Stos}
        \tab Programista, ma możliwość wykorzystania także stosu programowego, o głębokości 16 słów dwu bajtowych.

    \subsection{Pamięć eeprom}
        \tab Ostatnią przestrzenią do magazynowania danych użytkownika jest kość pamięci EEPROM o pojemności 256kB - co odpowiada 128 słowom bitowym.
    
    \newpage
    \begin{table}[!h]
        \centering
        % \vspace{2.5cm}
        \begin{tabularx}{\textwidth}{|c|L{5cm}|>{\centering\arraybackslash}X|>{\centering\arraybackslash}p{1.5cm}|}\hline
            Nr. & Instrukcja & Opis & Bajty\\\hline
             1. & NOP                   & Nic nie rób & 1\\\hline
             2. & END                   & Zakończ wykonywanie programu & 1\\\hline
             3. & SLEEP                 & Usypia procesor i wyłącza wszystkie peryferia* & 1 \\\hline
             4. & ADR\_SET $<reg>$      & Ustaw wskaźnik adresu pamięci & 1 \\\hline
             5. & I2C\_WRITE $<reg>$    & Zapisz do pamięci EEPROM wartość & 1 \\\hline
             6. & I2C\_READ $<reg>$     & Odczyt pamięci EEPROM & 1 \\\hline
             7. & INNER\_READ $<reg>$   & Odczyt pamięci programu & 1 \\\hline
             8. & LDR $<reg>, <val>$    & Zapisz do rejestru wartość & 3 \\\hline
             9. & JUMP\_IF $<reg>,\ <adr>$      & Skocz pod adres jeśli wartość z rejestru $\neq 0$ & 3 \\\hline
            10. & JUMP\_IF\_NOT $<reg>,\ <adr>$ & Skocz pod adres jeśli wartość z rejestru $= 0 $ & 3 \\\hline  
            11. & NOT $<reg>$           & Zaneguj wartość w rejestrze & 1 \\\hline
            12. & ADD $<reg1>,\ <reg2>$ & $reg1 + reg2 = reg1$ & 1 \\\hline
            13. & SUB $<reg1>,\ <reg2>$ & $reg1 - reg2 = reg1$ & 1 \\\hline
            14. & OR  $<reg1>,\ <reg2>$ & $reg1 \wedge  reg2 = reg1$ & 1 \\\hline
            15. & AND $<reg1>,\ <reg2>$ & $reg1 \vee  reg2 = reg1$ & 1 \\\hline
            16. & XOR $<reg1>,\ <reg2>$ & $reg1 \oplus  reg2 = reg1$ & 1 \\\hline
            17. & SHIFT\_LEFT $<reg>$   & Przesuń wartość w rejestrze w lewo o 1 & 1 \\\hline
            18. & SHIFT\_RIGHT $<reg>$  & Przesuń wartość w rejestrze w prawo o 1 & 1\\\hline
            19. & SHIFT\_8\_LEFT $<reg>$ & Przesuń wartość w rejestrze w lewo o 8 & 1 \\\hline
            20. & SHIFT\_8\_RIGHT $<reg>$& Przesuń wartość w rejestrze w prawo o 8 & 1\\\hline
            21. & PUSH $<reg>$          & Wrzuć wartość z rejestru na stos & 1 \\\hline
            22. & POP  $<reg>$          & Zdejmij wartość ze stosu i umieść w rejestrze & 1 \\\hline
            23. & JUMP $<adr>$          & Skod bezwzględnie pod adres & 3 \\\hline
            24. & CALL $<adr>$          & Skocz do funkcji\footnotemark & 3 \\\hline
            25. & RET  $<adr>$          & Wraca z funkcji & 1\\\hline
            26. & INC $<reg>$           & Inkrementuje wartość w rejestrze & 1 \\\hline
            27. & DEC $<reg>$           & Dekementuje wartość w rejestrze & 1 \\\hline
            28. & UART\_READ $<reg>$    & Przeczytaj znak wysłany po serialu & 1 \\\hline
            29. & UART\_SEND $<char>$   & Wyświetl znak z pamięci & 2 \\\hline
            30. & UART\_SEND $<reg>$    & Wyświetl wartość zapisaną w rejestrze & 1 \\\hline
            31. & LED $<reg>$           & Zapal/Zgaś diode & 1 \\\hline
            32. & WAIT $<reg>$          & Poczekaj czas wyrażony w ms & 1 \\\hline
            
            32. & FORWARD               & Uruchom silniki do przodu & 1\\\hline
            33. & BACKWARD              & Uruchom silniki w tył & 1 \\\hline
            34. & TURN $<reg>$        & Obróć pojazd o zadany kąt & 1 \\\hline
            35. & STOP                  & Zatrzymaj silniki & 1 \\\hline

            36. & MEASURE $<reg>$       & Dokonaj pomiaru odległości, wynik zapisz w rejestrze & 1 \\\hline
            37. & ROTATE $<reg>$     & Obróć głową o zadany kąt & 1 \\\hline
            % \multicolumn{4}{|c|}{Instrukcje dla akcelerometru}\\\hline
            % 32. 

        \end{tabularx}
        % \vspace{2.5cm}
        \caption{Tabela instrukcji „Azora”}
    \end{table}
    \footnotetext[1]{Zapisuje na stosie adres powrotu}