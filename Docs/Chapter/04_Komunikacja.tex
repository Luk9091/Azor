\section{Interfejs użytkownika}
    \tab Oprogramowanie pozwala na komunikację z „Azorem” oraz reprezentację danych pomiarowych.

    % Po uruchomieniu programu należy z wiersza poleceń wybrać port szeregowy, wykorzystywany do komunikacji przez moduł Bluetooth.
    

    \subsection{Podstawowe sterowanie „Azorem”}
        \tab Podstawowym sposobem sterowania jest interfejs graficzny, który można podzielić na 3 części:

        \subsubsection{Mapa skanowanego obszaru}        
            \tab Największą część okna aplikacji zajmuje mapa skanowanego obszaru, 
            na której czerwonymi liniami zaznaczone są przeszkody a niebieska strzałka reprezentuje „Azora”.

        \subsubsection{Radar}
            \tab Poniżej mapy po lewej stronie znajduje się radar. 
            Którego klikniecie wywołuje funkcję zbierania informacji na temat tego co widzi „Azor” oraz narysowanie ich na mapie. 
            Dodatkowo, na radarze widać aktualną pozycję „głowy Azora”.
        
        \subsubsection{Przyciski sterujące}
            \tab Ostatnią ale nie mniej ważną częścią interfejsu jest 6 strzałek i kulka. 
            Dzięki strzałkom góra/dół można poruszać „Azorem” w przód i w tył.
            Dwie kolejne strzałki w prawo i lewo pozwalają na obrót „Azora” o $90^\circ$.
            Ostatnie dwie strzałeczki odpowiadają za obrót głowy o $15^\circ$, a kuleczka pośrodku odpowiada za wykonanie pojedynczego pomiary odległości.


        % Okno interfejsu graficznego można podzielić na trzy obszary:
        % \begin{itemize}
        %     \item mapa skanowanego obszaru
        %     \item radar ukazujący obecne pole widzenia „Azora”
        %     \item przyciski do sterowania „Azorem”
        % \end{itemize}
        % \tab Interfejs graficzny posiada panel sterowania z przyciskami funkcyjnymi pozwalającymi na przemieszczanie 
        % i obracanie „Azora” oraz służące do obracania czujnika odległości i wykonywania pojedynczych pomiarów. 
        % Przyciski przód/tył umożliwiają przemieszczenie robota o około 100 mm w odpowiednim kierunku. Strzałki 
        % prawo/lewo pozwalają na jego obrót o $90^\circ$ w odpowiednią stronę, a zagięte strzałki w prawo/lewo 
        % umożliwiają obracanie czujnikiem odległości z krokiem $15^\circ$. Położenie „Azora” na mapie jest 
        % reprezentowane za pomocą niebieskiego kursora, a obrót „głową” jest pokazany na radarze za pomocą białego 
        % kursora. Wykonanie pojedynczego pomiaru umożliwia okrągły przycisk znajdujący się na środku panelu sterowania. 
        % Aby wykonać automatyczny cykl skanowania, należy kliknąć na radar. Spowoduje to automatyczne obracanie 
        % czujnikiem odległości z krokiem $3^\circ$ i zapisywanie danych pomiarowych.
    % \subsection{Reprezentacja danych}
    %     \tab Dzięki możliwości obracania czujnikiem odległości w zakresie $0^\circ-180^\circ$, „Azor” może skanować
    %     półkole o promieniu 720 mm. Wyniki pojedynczego cyklu skanowania są wyświetlane na radarze, znajdującym się
    %     z lewej strony interfejsu użytkownika.\\
    %     Na podstawie zebranych danych i obecnego położenia „Azora” zostaje nakreślona mapa obszaru.
    
    \newpage
    \subsection{Zaawansowane sterowanie}
        \tab Oprócz sterowania za pomocą przycisków istnieje możliwość wysyłania poleceń do „Azora” za pośrednictwem Command Line'a.\\
        Możliwe komendy:
        \begin{itemize}
            \item \textit{forward [wartość]}            - przemieszczenie „Azora” w przód o podaną odległość w mm, domyślna wartość to 100 mm,
            \item \textit{backward [wartość]}           - przemieszczenie „Azora” w tył o podaną odległość w mm, domyślna wartość to 100 mm,
            \item \textit{left [wartość]}               - obrót „Azora” w lewo o podany kąt, wartość domyślna to $90^\circ$,
            \item \textit{right [wartość]}              - obrót „Azora” w prawo o podany kąt, wartość domyślna to $90^\circ$,
            \item \textit{head left/right [wartość]}    - obrót czujnikiem odległości w lewo/prawo o podany kąt,wartość domyślna to $6^\circ$,
            \item \textit{head set [wartość]}           - obrót głową „Azora” do zadanego kąta z zakresu $0^\circ-180^\circ$, domyślnie $90^\circ$, co oznacza „patrzenie” w przód,
            \item \textit{head measure}                 - wykonanie pomiaru odległości dla obecnego ustawienia czujnika odległości,
            \item \textit{acc}                          - zwrócenie aktualnej wartości przyspieszenia na każdej z 3 osi,
            \item \textit{magnet}                       - zwrócenie zmierzone wartości pola magnetycznego na każdej z 3 osi,
            \item \textit{azimuth}                      - zwrócenie aktualnej wartości azymutu,
            \item \textit{distance}                     - zwrócenia wartości ostatnio pokonanej odległości,
            \item \textit{time}                         - zwrócenie czasu pracy silników,
            \item \textit{velocity}                     - zwrócenie wartości ostatnio zarejestrowanej prędkości,
            \item \textit{radar}                        - wykonanie automatycznego pomiaru odległości w zakresie $0^\circ-180^\circ$ z krokiem $3^\circ$,
            \item \textit{exit}                         - zakończenie działania programu,
        \end{itemize}
    