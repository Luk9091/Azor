\section{Metody pomiarowe}
    \subsection{Odległość -- dalmierz}
        \tab Podstawowym zadaniem „Azora” jest stworzenie mapy terenu.
        Proces ten wykonywany jest za pomocą dalmierza HC-SR04 zamocowanego na serwomechanizmie, dzięki czemu dalmierz może obracać się w osi Z od $0^\circ$ do $180^\circ$.\\
        Schemat połączenia:
        \begin{figure}[!h]
            \centering
            \begin{circuitikz}
                \draw
                    (0, 0) -- (0, -5)
                    (-2, 0) node[]{$\mu P$}

                    (3, -1.75) node[draw, rectangle, minimum width = 2cm, minimum height = 1cm](HC){HC-SR04}
                    (3, -4) node[draw, rectangle, minimum width = 2cm, minimum height = 1cm](SG){SG-90}

                    (0, -1.5) coordinate(ECHO) node[left]{(ECHO) PD4}
                    (0, -2.0) coordinate(TRIG) node[left]{(TRIG) PB6}
                    (0, -4.0) coordinate(PWM)  node[left]{(PWM) PB3}

                    (ECHO) -- ++ (2, 0)
                    (TRIG) -- ++ (2, 0)

                    (PWM) -- ++ (2, 0)

                    (HC) ++ (0, 0.5) node[vcc]{$V_{CC}$}
                    (SG) ++ (0, 0.5) node[vcc]{$V_{CC}$}

                    (HC) --++ (1.5, 0) node[ground]{}
                    (SG) --++ (1.5, 0) node[ground]{}
                ;
            \end{circuitikz}
            \caption{Schemat podłączenia czujnika odległości i serwomechanizmu}
        \end{figure}\\
        Poniżej przedstawiono algorytm pomiaru odległości:
        \begin{figure}[!h]
            \centering
            \begin{circuitikz}
                \draw
                    (0, 0) node   [draw, circle,  minimum width = 2.5cm, text width = 2cm, align=center](start){Wysłanie sygnału TRIG}
                    (6, 0) node   [draw, diamond, minimum width = 2.5cm, text width = 2cm, align=center](if1){Czy odebrano ECHO}
                    (12, -2.5) node [draw, circle,  minimum width = 2.5cm, text width = 2.5cm, align=center](timer){Rozpoczęcie pomiaru czasu}
                    (6, -5) node  [draw, diamond, minimum width = 2.5cm, text width = 2cm, align=center](if2){Czy cały sygnał wrócił}
                    (0, -5) node  [draw, circle,  minimum width = 2.5cm, text width = 2cm, align=center](end){Zwrócenie odległości}
                ;
                \draw[-stealth, thick, line width = 0.5mm] (start) -- (if1);
                \draw[-stealth, thick, line width = 0.5mm] (if1.east) to[bend left] node[above]{Tak} (timer.north);
                \draw[-stealth, thick, line width = 0.5mm] (timer.south) to[bend left] (if2.east);
                \draw[-stealth, thick, line width = 0.5mm] (if2) -- node[below]{Tak} (end);

                % \path(start) edge[loop above];
                \draw[-stealth, thick, line width = 0.5mm] (if1) to[loop right, rotate = 80] node[] {Nie} ();
                \draw[-stealth, thick, line width = 0.5mm] (if2) to[loop right, rotate = 80] node[] {Nie} ();

            \end{circuitikz}
            \caption{Schemat blokowy algorytmu pomiaru odległości}
        \end{figure}

    \subsection{Prędkość -- enkoder}\label{section:measure:enkoder}
        \tab Pomiar prędkości, a właściwie odległości odbywa się za pomocą enkodera zamocowanego do jednego z silników.
        Kolejne impulsy enkodera zliczane są za pomocą Timera 0 ustawionego w tryb normalnej pracy z zewnętrznym źródłem taktującym.  % 
        % \begin{figure}[!h]
        %     \centering
        %     \begin{tikzpicture}
        %         \draw
        %             (0, 1) --++ (0, -2)
        %             (0, 0) node[left]{(T0) PD4} --++(1, 0) node[draw, rectangle, right](enkoder){enkoder}
        %             % (enkoder.north) node[vcc]{$V_{CC}$}
        %             (enkoder.south) node[ground]{}
        %         ;
        %     \end{tikzpicture}
        %     \caption{Schemat podłączenia enkodera}
        % \end{figure}
        Odebrane impulsy mnożone są przez doświadczalnie wyznaczony mnożnik i w ten sposób uzyskiwana jest przebyta droga.
        \begin{equation}
            \text{Współczynnik} = 6.3 \frac{mm}{tic}
        \end{equation}
        Jednocześnie podczas pracy silników mierzony jest czas, dzięki czemu można określić prędkość „Azora” zgodnie ze wzorem:
        \begin{equation}
            v = \frac{x\cdot 6.3}{\frac{1}{8}\cdot t}\frac{[mm]}{[us]}
        \end{equation}
        gdzie:
        \begin{itemize}
            \item $x$ -- ilość zliczonych impulsów z enkodera,
            \item $t$ -- czas zmierzony przez timer 1.
        \end{itemize}
        
    \subsection{Przyspieszenie -- akcelerometr}
        \tab Pomiaru prędkości można dokonać także w inny sposób.
        „Azor”, wyposażony jest w cyfrowy akcelerometr (MMA8451) komunikujący się z jednostką sterującą za pomocą I²C.
        Dzięki akcelerometrowi można wykryć czy „Azor” nie został podniesiony -- za każdym razem gdy „Azor” zostanie ruszony, zgłaszane jest zewnętrzne przerwanie. 
        Jednocześnie pomiar przyspieszenia pozwala z dużym przybliżeniem określić aktualną prędkość „Azora”.\\
        Zgodnie ze wzorem:
        \begin{equation}
            v = \int\limits_{0}^{t} a dt \approx \sum_{i = 0}^{N} a_i \cdot t_i
        \end{equation}
        \begin{itemize}
            \item $N$ -- liczba próbek
            \item $a_i$ -- kolejne próbki przyspieszenia
            \item $t_i$ -- kolejne zapisane wartości czasu
        \end{itemize}

    \subsection{Magnetometr -- cyfrowy kompas}\label{section:measure:azimuth}
        \tab Czujnik $QMC588L$ jest 3-osiowym czujnikiem pola magnetycznego. Zwraca on 16-bitowe wartości zmierzone na każdej z osi, za pomocą magistrali I²C.
        
        Pomiar azymutu - odchylenia od północy, jest prostym sposobem wyznaczenie obrotu „Azora” wzdłuż własnej osi.
        Wartości zmierzone po odebraniu przez kontroler muszą zostać przeskalowane w celu eliminacji zakłóceń.
        Po czym tak zmodyfikowany pomiar trafia do  funkcji trygonometrycznej:
        \begin{equation}
            \text{azymut} = arctan2(y, x)
        \end{equation}
        Tak otrzymana wartość azymutu zawiera się w wartościach od $-180^\circ$ do $180^\circ$, aby otrzymać ostateczny wynik należy do wartości ujemnych dodać $360^\circ$.
