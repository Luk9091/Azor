\section{Metody pomiarowe}
    \subsection{Odległość -- dalmierz}
        \tab Podstawowym zadaniem „Azora” jest stworzenie mapy tereny.
        Proces ten wykonywany jest za pomocą dalmierza HC-SR04 zamocowane na serwo mechanizmie, dzięki czemu dalmierz może obracać się w osi Z od $0^\circ$ do $180^\circ$.\\
        Schemat połączenia:
        \begin{figure}[!h]
            \centering
            \begin{circuitikz}
                \draw
                    (0, 0) -- (0, -5)
                    (-2, 0) node[]{$\mu P$}

                    (3, -1.75) node[draw, rectangle, minimum width = 2cm, minimum height = 1cm](HC){HC-SR04}
                    (3, -4) node[draw, rectangle, minimum width = 2cm, minimum height = 1cm](SG){SG-90}

                    (0, -1.5) coordinate(ECHO) node[left]{(ECHO) PD4}
                    (0, -2.0) coordinate(TRIG) node[left]{(TRIG) PB6}
                    (0, -4.0) coordinate(PWM)  node[left]{(PWM) PB3}

                    (ECHO) -- ++ (2, 0)
                    (TRIG) -- ++ (2, 0)

                    (PWM) -- ++ (2, 0)

                    (HC) ++ (0, 0.5) node[vcc]{$V_{CC}$}
                    (SG) ++ (0, 0.5) node[vcc]{$V_{CC}$}

                    (HC) ++ (1, 0) node[ground]{}
                ;
            \end{circuitikz}
        \end{figure}

        \subsection{Algorytm działania:}
            \begin{figure}[!h]
                \centering
                \begin{circuitikz}
                    \draw
                        (0, 0) node [draw, circle, minimum width = 2cm, text width = 2cm, align=center]{Wysłanie sygnału TRIG}
                        (0, -4) node[draw, diamond, minimum width= 2cm, text width = 2cm, align=center]{Czy został odebrany sygnał z ECHO}
                    ;
                \end{circuitikz}
            \end{figure}

    \subsection{Prędkość -- enkoder}
    \subsection{Przyspieszenie -- akcelerometr}
    \subsection{Magnetometr -- cyfrowy kompas}