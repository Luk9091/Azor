\section{Struktura katalogów i kodu}
    \tab Poniższy rozdział zostanie podzielony na dwie współbieżne sekcje:
    Opis struktury katalogów oraz opis kodu.

    % \subsection{Struktura katalogów}
    \paragraph{Struktura katalogów:\\}
        Projekt ten zawiera się w kilku katalogach, każdy przeznaczony na inną część projektu.
        \begin{itemize}
            \item Assembler,
            \item Docs,
            \item GUI,
            \item Src.
        \end{itemize}

    \subsection{Assembler}
        \tab W powyższym folderze zawarte są pliki do kompilacji programów użytkownika.

        Plik $compile.py$ zawiera skrypt odpowiedzialny za kompilację programu użytkownika.
        Lwią część skryptu zajmuje słownik, z listą instrukcji jako klucze i odpowiadającymi im kodami. 
        Dodatkowo w wartościach słownika przechowywane są informację na temat liczby wykorzystywanych rejestrów oraz ilości dodatkowych bajtów wymaganych przez instrukcję.

        Drugim składnikiem tego katalogu jest skrypt $programming.py$ odpowiedzialny za przeczytanie „skompilowanego” pliku oraz wgranie go do pamięci EEPROM mikrokontrolera.

    \subsection{Docs}
        \tab Drugim w kolejności jest katalog $Docs$, który zawiera kod poniższej dokumentacji.
    
    \subsection{GUI}
        \tab Następnym katalogiem jest folder z graficznym interfejsem użytkownika.
        \begin{itemize}
            \item arrow.py      -- skrypt do rysowania strzałek w GUI,
            \item azor.py       -- skrypt do komunikacji z „Azorem”,
            \item circle.py     -- skrypt do rysowania kulki,
            \item commands.py   -- skrypt do interpretacji komend z CLI,
            \item geometry.py   -- data class z geometrią, poszczególne obiekty dziedziczą tę klasę i w jej obiektach przechowują swoje wymiary i położenie,
            \item GUI.py        -- skrypt zbiorczy do narysowania i rozłożenia obiektów na planszy,
            \item main.py       -- skrypt główny, odpowiedzialny za połączenie CLI i GUI,
            \item map.py        -- skrypt odpowiedzialny za rysowanie mapy tereny,
            \item radar.py      -- skrypt rysowania radaru,
            \item simulation.py -- skrypt zawierający symulację „Azora” w przypadku gdy użytkownik nie ma możliwości połączenia się z oryginałem.
        \end{itemize}
% 
        Dodatkowo w tym katalogu znajduje się jeszcze jeden katalog $Font/$ zawierający font używany w GUI.


    \subsection{Src}
        \tab Ostatnim katalogiem jest $Src$, zawierający wszystkie pliki z programem głównym „Azora”. 
        \begin{itemize}
            \item accelerometer*-- obsługa akcelerometru,
            \item coding*       -- switch z programowaniem i wykonywaniem kodu użytkownika,
            \item compass*      -- obsługa kompasu,
            \item eeprom*       -- obsługa pamięci wewnętrznej i zewnętrznej pamięci EEPROM,
            \item engine*       -- obsługa silników,
            \item I2C*          -- biblioteka dodatkowa do I2C,
            \item main          -- główna pętla programu,
            \item PWM*          -- biblioteka pomocnicza do sterowania obrotem „głowy Azora” ,
            \item sonic*        -- obsługa dalmierza - współpracuje z timerem 1,
            \item timer*        -- biblioteka pomocnicza do obsługi timerów i liczników,
            \item uart*         -- biblioteka do komunikacji po USART,
        \end{itemize}
        \vspace{4pt}
        \hrule
        \vspace{16pt}
        \tab * plik podzielony na header $(.hpp)$ i plik źródłowy $(.cpp)$.

    

    % \subsection{Peryferia}
        